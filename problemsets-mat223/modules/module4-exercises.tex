\begin{exercises}
	% Topics:
	% dot products, orthogonality
	% unit vectors, vector length,
	% angle between vectors, distance between vectors
	% lines, planes, hyperplanes in normal form
	\begin{problist}
		\prob  Compute the following dot products.
		\begin{enumerate}
			\item   $\mat{9\\4} \cdot \mat{10\\-3}$
			\item   $\mat{1\\36\\2} \cdot \mat{0\\0\\1}$
			\item   $\mat{7\\6\\-3} \cdot \left(\mat{5\\11\\-1} + \mat{-2\\-6\\-1}\right)$
			\item   $\mat{1\\3\\0\\-5\\5} \cdot \mat{1\\2\\2\\1\\2}$
			\item   $\left(\frac{1}{2}\mat{2\\5\\4}\right) \cdot \mat{1\\0\\-1}$
		\end{enumerate}
		\begin{solution}
			\begin{enumerate*}
				\item 78 
				\item 2 
				\item 57 
				\item 12 
				\item -1 
			\end{enumerate*}
		\end{solution}
		\prob Compute the length of the following vectors.
		\begin{enumerate}
			\item $\mat{2\\0}$
			\item $\mat{1\\2\\3}$
			\item $4\mat{5\\-6\\15\\2}$
		\end{enumerate}
		\begin{solution}
			\begin{enumerate*}
				\item 2 
				\item $\sqrt{14}$ 
				\item $4\sqrt{290}$ 
			\end{enumerate*}
		\end{solution}
		\prob For each pair of vectors listed below, determine if the angle between the vectors is greater than,
		less than, or equal to $90^\circ$.
		\begin{enumerate}
			\item $\mat{1\\0}$ and $\mat{-3\\4}$
			\item $\mat{1\\0\\1}$ and $\mat{-5\\4\\-3}$
			\item $\mat{1\\2\\3}$ and $\mat{-1\\-1\\2}$
		\end{enumerate}
		\begin{solution}
			\begin{enumerate*}
				\item Greater than $90^\circ$
				\item Greater than $90^\circ$
				\item Less than $90^\circ$
			\end{enumerate*}
		\end{solution}
		\prob For each vector, find two \emph{unit} vectors orthogonal to it.
		\begin{enumerate}
			\item $\mat{0\\1}$
			\item $\mat{1\\2}$
			\item $\mat{1\\3\\5}$
			\item $\mat{-13\\-4\\5}$
			\item $\mat{0\\1\\1\\\frac{1}{2}}$
		\end{enumerate}
		\begin{solution}
			\begin{enumerate*}
				\item $\mat{1\\0}$ and $\mat{-1\\0}$
				\item $\dfrac{1}{\sqrt{5}}\mat{2\\-1}$ and $\dfrac{1}{\sqrt{5}}\mat{-2\\1}$
				\item $\dfrac{1}{\sqrt{10}}\mat{3\\-1\\0}$ and $\dfrac{1}{\sqrt{26}}\mat{5\\0\\-1}$
				\item $\dfrac{1}{\sqrt{6}}\mat{1\\-2\\1}$ and $\dfrac{1}{\sqrt{41}}\mat{0\\5\\4}$
				\item $\mat{1\\0\\0\\0}$ and $\dfrac{1}{\sqrt{2}}\mat{0\\1\\-1\\0}$
			\end{enumerate*}
		\end{solution}
		
		\prob Compute the distance between the following pairs of vectors.
		\begin{enumerate}
			\item $\mat{-1\\1}$ and $\mat{-1\\-4}$
			\item $\mat{2\\-6\\5}$ and $\mat{-4\\7\\-3}$
			\item $\mat{1\\1\\1}$ and $\mat{-1\\-1\\-1}$
			\item $\mat{0\\0\\0\\0\\0}$ and $\mat{1\\1\\1\\1\\1}$
		\end{enumerate}
		\begin{solution}
			\begin{enumerate*}
				\item $5$ 
				\item $\sqrt{269}$ 
				\item $2\sqrt{3}$ 
				\item $\sqrt{5}$ 
			\end{enumerate*}
		\end{solution}
		
		\prob
		\begin{enumerate}
			\item Which vector out of $\mat{1\\0}$, $\mat{0\\1}$, and $\mat{4\\1}$
			has a direction closest to that of $\mat{3\\5}$?
			\item Which vector out of $\mat{2\\3\\4}$, $\mat{1\\-1\\-1}$, and $\mat{-3\\0\\1}$
			has a direction closest to that of $\mat{1\\0\\1}$?
		\end{enumerate}
		\begin{solution}
			\begin{enumerate*}
				\item $\mat{0\\1}$ 
				\item $\mat{2\\3\\4}$
			\end{enumerate*}
		\end{solution}
		
		\prob For each plane specified, express the plane in both vector form and normal form.
		\begin{enumerate}
			\item The plane $\mathcal P\subseteq\R^3$ passing through the points
			$A=(2,0,0)$, $B=(0,3,0)$ and $C=(0,0,-1)$.
			\item The plane $\mathcal Q\subseteq\R^3$ passing through the points
			$D=(1,1,1)$, $E=(1,-2,1)$ and $F=(0,12,0)$.
		\end{enumerate}
		\begin{solution}
			\begin{enumerate}
				\item $\vec x=t\mat{2\\0\\1}+s\mat{0\\3\\1}+\mat{2\\0\\0}$ and $\mat{3\\2\\-6}\cdot\left(\vec x-\mat{2\\0\\0}\right)=0$
				\item $\vec x=t\mat{1\\1\\1}+s\mat{0\\1\\0}$; $\mat{1\\0\\-1}\cdot\vec x=0$
			\end{enumerate}
		\end{solution}
		
		\prob
		\begin{enumerate}
			\item Let $\mathcal A\subseteq \R^3$ be the plane passing through $\mat{0\\1\\1}$
			and with normal vector $\mat{-1\\-1\\-1}$. Write $\mathcal A$ in vector form.
			\item Let $\mathcal B\subseteq \R^3$ be the plane passing through $\mat{1\\2\\3}$
			and with normal vector $\mat{1\\-1\\0}$. Write $\mathcal B$ in vector form.
		\end{enumerate}
		\begin{solution}
			\begin{enumerate*} 
				\item $\vec x = t\mat{1\\1\\0}+s\mat{0\\1\\1}$
				\item $\vec x = t\mat{1\\1\\0}+s\mat{0\\0\\1}+\mat{1\\2\\3}$ 
			\end{enumerate*}
		\end{solution}
		
		\prob In this problem we will prove some algebraic properties of the dot product.
		\begin{enumerate}
			
			\item Show by direct computation
			\[
			\left(\mat{1\\2} + \mat{3\\4}\right) \cdot \mat{5\\6} =
			\mat{1\\2} \cdot \mat{5\\6} + \mat{3\\4} \cdot \mat{5\\6}
			\]
			\item
			For $\vec x,\vec y,\vec z\in \R^2$, justify whether or not it always holds that
			\[
			(\vec x + \vec y) \cdot \vec z = \vec x \cdot \vec z + \vec y \cdot \vec z.
			\]
			Does the same conclusion hold true when $\vec x,\vec y,\vec z\in \R^n$?
			\item Show by direct computation
			\[
			\left(6\mat{2\\3}\right) \cdot \mat{4\\5} =
			6\left(\mat{2\\3} \cdot \mat{4\\5}\right)
			\]
			\item
			For $\vec x,\vec y\in \R^2$ and $k \in \R$, Justify whether or not it always holds that
			\[
			(k\vec x) \cdot \vec y = k(\vec x \cdot \vec y).
			\]
			Does the same conclusion hold true when $\vec x,\vec y\in \R^n$?
			
			\item The dot product is called \emph{distributive}.
			Is this a good word to describe the dot product? Why?
		\end{enumerate}
		\begin{solution}
			\begin{enumerate}
				\item We get the value $56$ on both sides.
				
				\item Writing the vectors in terms of the coefficients, we get
				\[(x_1+y_1)z_1+(x_2+y_2)z_2=x_1z_1+y_1z_1+x_2z_2+y_2z_2\]
				which is always true.
				
				In the general case, the left side will be the sum of $(x_i+y_i)z_i$ and the right side 
				will be the sum of $x_iz_i+y_iz_i$. Since these two terms are always equal, the two sums 
				are equal. So yes, the same conclusion hold true in all dimensions.
				
				\item We get the value $138$ on both sides.
				
				\item Writing it in terms of the coefficients, we get
				\[(kx_1)y_1+(kx_2)y_2=k(x_1y_1+x_2y_2)\]
				which is always true.
				
				In the general case, the left side will be the sum of $kx_iy_i$ and the right side will 
				be the product of $k$ and the sum of $x_iy_i$. Distributing the product of $k$ onto the 
				sum, we get that these two results are equal. So yes, the same conclusion hold true in 
				all dimensions.
				
				\item The dot product distributes onto sums, just like the typical multiplication of 
				real numbers.
			\end{enumerate}
		\end{solution}
	
        	\prob Let $\vec u, \vec v \in \R^{n}$. In this problem, we will prove
		$\Abs{\vec u \cdot \vec v}\leq \Norm{\vec u}\Norm{\vec v}$. This is called
		the \emph{Cauchy-Schwarz inequality}.
		\begin{enumerate}
			\item Assuming the geometric definition
				$\vec u \cdot \vec v = \Norm{\vec u}\Norm{\vec v}\cos \theta$
				where $\theta$ is the angle between $\vec u$ and $\vec v$, prove
				the \emph{Cauchy-Schwarz inequality}.

			\item The \emph{Cauchy-Schwartz inequality} can also be proved using
				\emph{only} the algebraic definition of the dot product. Keep in
				mind the following facts which come from the algebraic definition:

				\begin{enumerate}
					\item Explain why the result is immediate if one (or both)
						of $\vec u, \vec v$ is the zero vector.

					\item Assume $\vec u, \vec v$ are non-zero vectors. Consider
						the function $P: \R \to \R$ where $P(t) = \Norm
						{t\vec u - \vec v}^{2}$. Convince yourself that
						$P(t) \geq 0$ for all $t \in \R$.

					\item Simplify $P(t)$ into a quadratic formula so that
						$P(t) = at^{2}- bt + c$ by determining its coefficients
						$a, b, c$ in terms of $\vec u, \vec v$.

					\item Prove
						$\Abs{\vec u \cdot \vec v}\leq \Norm{\vec u}\Norm
						{\vec v}$, the \emph{Cauchy-Schwarz inequality}. Hint: Consider
						$P(\frac{b}{2a})$.

					\item Prove
						$\Abs{\vec u \cdot \vec v}= \Norm{\vec u}\Norm{\vec v}$
						if the vectors $\vec u, \vec v$ are scalar multiples of
						each other.
				\end{enumerate}
		\end{enumerate}

		\prob Let $\vec u, \vec v \in \R^{n}$. In this problem, we will prove
		$\Norm{\vec u + \vec v}\leq \Norm{\vec u}+ \Norm{\vec v}$. This is called the
		\emph{Triangle Inequality}.
		\begin{enumerate}
			\item Draw a picture and explain why it is called the \emph{Triangle Inequality}.

			\item Express $\Norm{\vec u + \vec v}$ in terms of dot products and
				square roots.

			\item Prove the \emph{Triangle Inequality}.
		\end{enumerate}

		\prob Let $\mathcal{A}= \{\vec v_{1},\dots , \vec v_{k}\} \subset \R^{n}$.
		$A$ is a set of mutually orthogonal vectors if for all $i \neq j$, we have
		$\vec v_{i}\cdot \vec v_{j}= 0$.
		\begin{enumerate}
			\item Suppose $\mathcal{A}$ is a set of mutually orthogonal vectors.
				Is $\mathcal{A}$ a linearly independent set? Why or Why not?

			\item Suppose $\mathcal{A}$ is a set of mutually orthogonal and non-zero
				vectors. Is $\mathcal{A}$ a linearly independent set? Why or Why
				not?
		\end{enumerate}
	\end{problist}
\end{exercises}
